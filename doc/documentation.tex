% File name: documentation.tex
% Date:      13.12.2011 16:13
% Authors:   Radek Fér          <xferra00@stud.fit.vutbr.cz>
%            Miroslav Paulík    <xpauli00@stud.fit.vutbr.cz>

\documentclass[a4paper,12pt,titlepage]{article}
\usepackage[czech]{babel}
\usepackage{color}
\usepackage{tabularx}
\usepackage{hyperref}
\usepackage[utf8]{inputenc}
\usepackage[left=2cm, top=3cm, text={17cm, 24cm}]{geometry}
\usepackage{graphicx} % for .eps images
\usepackage{fancyvrb,fancybox,calc}
\usepackage[svgnames]{xcolor}
 
\hypersetup{linktoc=all}
%\hypersetup{pdfborder={0 0 0 [0 0]}
\hypersetup{colorlinks=true}
\hypersetup{linkcolor=blue}

% European layout (no extra space after `.')
\frenchspacing

% no indent, free space between paragraphs
\setlength{\parindent}{0pt}
\setlength{\parskip}{1ex plus 0.5ex minus 0.2ex}

\begin{document}
\renewcommand{\refname}{Literatura}

\title{\LARGE Pokus o vytvoření jednoduchého 3D \uv{myšítka} \\
       {\large Projektová dokumentace do předmětu ITU 2011/12}}
\author{ \begin{tabularx}{\textwidth}{X r l X}
& Radek Fér & \texttt{xferra00@stud.fit.vutbr.cz} & \\
& Miroslav Paulík & \texttt{xpauli00@stud.fit.vutbr.cz} & \\
\end{tabularx}
}
\date{\today, FIT VUT Brno}

\maketitle

\newpage

\thispagestyle{empty}
\tableofcontents
\newpage
\setcounter{page}{1}

% \label{summary}
% \cite{citId}
% \url{www.google.com}

%\begin{figure}[htb]
%\centering
%\includegraphics[width=0.8\textwidth]{img/myplot.eps}
%\caption{Mycaption}
%\label{fig:myplot}
%\end{figure}

\section{Úvod}\label{intro}
Tento projekt si klade za cíl odzkoušet možnosti využití speciálního
vstupního zařízení (zkonstruovaného podle myšlenky jednoho z autorů),
které {\bf využívá 2 dostupná 2D polohovací zařízení (myš, touchpad, \ldots)
k vytvoření jednoho 3D polohovacího zařízení}. A to tak, že pohyby
oněch dvou 2D polohovacích zařízení jsou slučovány do pohybů nad 3D.

Dále si projekt klade za cíl navrhnout a otestovat různá uživatelská rozhraní
využívající tohoto zařízení tak, aby se práce s virtuálním 3D prostorem
stala více efektivnější, intuitivnější a příjemnější.

% File name: navrh.tex
% Date:      13.12.2011 17:00

\section{Návrh}

Zařízení je navržené tak, aby jeho služeb mohl využívat úplně každý.
Hardwarové vybavení (tedy de facto pořizovací cena zařízení\,--\,software
je šířen v otevřené podobě pod GPL licencí) obnáší pouze 2 klasické myši.

\subsection{Princip činnosti}

Princip činnosti navrhovaného zařízení je zobrazen na obrázku \ref{fig:basicidea}.

\begin{figure}[htb]
\centering
\includegraphics[width=0.6\textwidth]{img/basic_idea.eps}
\caption{Princip činnosti zařízení}
\label{fig:basicidea}
\end{figure}

Nejdůležitější část zařízení tvoří část {\bf Merging}, kde dochází k mapování
čtyřech vstupních os (osy $x$ a $y$, u obou myší) na osy 3 (odpovídající
osám 3D prostoru, ve kterém chceme pracovat). Implementace této části je
detailně popsána v sekci \ref{merging}.


\subsection{Fyzické uspořádání}

Při návrhu zařízení jsme zohlednili praktickou stránku, jak zacházet
se 2 myšmi:
\begin{itemize}
\item{{\bf Vše v 1 fyzické rovině}}
    \begin{itemize}
    \item[$+$]{ obě ruce stejně zatížené, žádná nemusí držet váhu myši a sebe u vertikálně položené podložky}
    \item[$-$]{ ne tak intuitivní, jako následující uspořádání}
    \end{itemize}
\item{{\bf Podložky obou myší fyzicky umístěny ortogonálně}\,--\,vlastnosti viz předchozí uspořádání, ale naopak}
\item{{\bf Varianta pro 1 ruku}\,--\,na klasické myši je zleva (myš pro praváky) umístěn drobný touchpad ovládaný palcem ovládající zbývající osu}
\end{itemize}


\subsection{Existující zařízení}\label{solutions}

Obecný termín pro takováto zařízení je {\bf 3D myš}.

Na trhu se již dlouho vyskytují výrobky s podobnou funkcionalitou.
Jde např. od výrobky fy {\it 3Dconnexion}, kupř. {\bf SpaceNavigator}\,--\,zařízení
určené pro pohyb a minupulaci s objekty ve 3D. Toto zařízení umožňuje
ovládat 6 směrů pomocí jedné ruky (tažením, rotací, stlačením, vytažením
apod.).

Další možnosti:
\begin{itemize}
\item{výrobky fy {\bf Axsotic}\,--\,něco podobného, využívá standardu HID}
\item{{\bf Wii Remote Nunchuk}\,--\,obsahuje 3D akcelerometr a tlačítka}
\item{jistě by se našly i další}
\end{itemize}

% File name: implementace.tex
% Date:      13.12.2011 17:00

\section{Implementace}

\subsection{Overview}

\subsection{Merging}\label{merging}


\subsection{Knihovny}
GLUT, SDL

\subsection{Omezení naší implementace}

- Pouze Linux



% File name: experimenty.tex
% Date:      13.12.2011 17:00

\section{Experimenty a testování}
Cílem experimentální části je získání užitečných informací o testovaném subjektu a z nich vycházející další vývoj. Tato kapitola bude tedy popisovat stategii testování a experimentů na rozhraní m2m, výběr vhodné skupiny účastníků těchto pokusů a popisem jednotlivých experimentů a analýzy jejich výsledků.

\subsection{Strategie experimentování a testování}
Úplně jako první bylo potřeba zjistit, jakým způsobem by měla být aplikace ovládána. Které tlačítka obou polohovacích zařízení používat ke kterým činnostem a jak správně nastavit osy pohybu každého ze vstupních zařízení. Jelikož má rozhraní pracovat v prostředí 3D, musí existovat prostředek pro prostorovou orientaci. Cílem druhého experimentu se tedy stalo zjištění reakcí uživatelů na různé možnosti natáčení 3D scény s ohledem na pohyb v 3D prostoru a jeho přizpůsobení. Získání výše zmíněných informací umožnilo vytvořit několik demonstračních aplikací, jež se později staly instruktážními nástroji pro poslední experiment. Než k němu ale došlo, bylo mezi účastníky testování provedena krítká analýza uživatelského rozhraní pro odstranění drobných nedostatků těchto jednoduchých aplikací. Cílem celého procesu experimentování a testování však bylo hlavně získání údajů o míře efektivnosti práce s takovým rozhraním. A právě to bylo námplní závěrečného experimentu. 

\subsubsection{Výběr účastníků experimentů}
Největší vliv na výběr skupiny testujících uživatelů byla jejich schopnost pracovat s počítačem, uživatel mající problém obsluhovat jedno polohovací zařízení bude jistě mít problém i se zařízením druhým. Proto jsme požádali o účast v experimentování skupinu 15-ti vysokoškolských studentů, kteří s požadavkem běžného ovládání počítače nemají problém.

\subsubsection{experiment č. 1 - orientace v prostoru}
Jak již bylo uvedeno výše, cílem prvního experimentu bylo nalezení optimálního ovládání a zjištění dalších možných nastavení, které by si eventuálně uživatel mohl explicitně zvolit. Jako forma získání informací byl zvolen dotazník %priloha dotaznik 1
. Z analýzy jeho výsledků vyplynulo, že: %dodelat

Pro další vývoj tedy bylo potřeba naimplementovat možnost záměny souřadných os mezi polohovacími zařízeními navzájem. Za zmínku také stojí fakt, že jak pravorukým, tak levorukým vyhovovalo maximálně prohození souřadných os, ovšem prohození významu levého a pravého tlačítka bylo vnímáno jako nepotřebné%opravdu?
Dotazník nakonec posloužil i k orientačnímu zisku informací o dalších aktivitách a zkušenostech účastníků v oblasti práce v 3D aplikacích.

experiment1 - pohyb v prostoru - zjistit pozadavky na ovladani
experiment2 - orientace v 3d prostoru
testování   - odstraneni drobnych nedostatku
experiment3 - schopnost prizpusobit se novemu prostredi za ucelem odhadu efektivity prace ve 3d prostoru pomoci rozhrani m2m



\newpage

\appendix

\section{Návrh zadání}\label{zadani}

{\it Název:}~{\bf Pokus o vytvoření jednoduchého 3D polohovacího zařízení.}
\begin{itemize}
\item{2 myši (nebo jiné polohovací zařízení), každá ovládá pozici na jedné
      rovině ($xy$, $xz$ či $yz$).}
\item{Hlavní fór by byl v tom, že by i druhá myš (a ne jen kurzor v rovině kolmé
      k desce stolu) fyzicky jezdila po rovině kolmé k desce stolu.}
\item{Vhodnou kombinací vstupů z těchto 2 myší lze určit bod v 3D a případně
      s ním manipulovat (pozice na ose společné pro obě použité roviny
      by se určila jako průměr souřadnic).}
\item{Možné módy (možno měnit např. kombinací stisků tlačítek na myši):}
    \begin{itemize}
    \item{{\bf základní}\,--\,posunování 3D kurzoru v prostoru,
            tlačítkem se provede výběr nejbližšího objektu}
    \item{{\bf rotační}\,--\,rotace vybraného objektu pomocí např. pomocí
            \uv{circullar-scrolling}}
    \item{{\bf morfní}\,--\,změna nějakého atributu objektu
            (velikost, barva, tvar, ...)}
    \end{itemize}
\item{Nebo celé úplně jinak, cílem by bylo zkoumat schopnosti interakce
        takovéhoto HW zařízení.}
\item{K demonstraci by sloužila např. hra 3D piškvorky (pro jednoduchost
        by bylo implementováno pouze označování políček a ne logika
        hry\,--\,cílem projektu by nebylo vytvořit hru 3D piškvorky,
        ale \uv{prověření zařízení pro práci ve 3D}).}
\item{{\it openGL}}
\item{Možné využití (jen co mě napadlo): modelování 3D scény, hry,
        ovládání jeřábu, \ldots}
\end{itemize}

% \bibliographystyle{plain}
% \bibliography{itu}

\end{document}
