% File name: implementace.tex
% Date:      13.12.2011 17:00

\section{Implementace}

Součástí projektu je implementace ovladače zařízení a dále několika
demonstračních aplikací. Kód je psán jazykem C (slučovač) a C++ (demo aplikace).
Slučovač je nyní možné přeložit pouze v OS GNU/Linux,
díky silné a nutné vazbě na jeho vstupní subsystém. Demonstrační aplikace
(a tedy jakýkoli klientský software, který by chtěl využívat naše rozraní)
už přenositelné jsou (je zde pouze závislost na knihovnách SDL a GLUT).

\subsection{Slučovač}\label{merging_implem}

Jak již bylo zmíněno, implementace slučovače je silně vázána na vstupní
subsystém Linuxového jádra. V této části jsou tyto vazby popsány spolu
s tím, jak přesně slučovač funguje.

\subsubsection*{Vstupní sybsystém v Linuxu}
Každé vstupní zařízení rozpoznané Linuxem má v adresáři {\tt /dev/input/}
vlastní speciální soubor, který je možný použít z userspace k přístupu
k tomtuto zařízení, typicky pro čtení událostí.
Všechny události mají společný formát a jejich popis je možné nalézt
v {\tt /usr/include/linux/input.h}.

Nás zajímají události o pohybu myši, jejichž typ je {\tt EV\_REL}
a dále se dělí na zprávy {\tt REL\_X} a {\tt REL\_Y}.
Každá takováto zpráva nese navíc časové razítko a velikost změny
polohy v dané ose.

Upozornění: Xka (Xorg server) si většinou myš, která je nastavena
jako {\it CorePointer} zaberou pro sebe a ta je poté pro další čtení
\uv{němá}.

Slučovač tedy nedělá nic jiného, než že čte ze dvou souborů (např.
{\tt /dev/input/event10} a {\tt /dev/input/event11}), tyto data
interpretuje jako události a hodnoty výchylek os mapuje podle {\it mapstringu}
na výchylky os výstupních.

\subsubsection*{Virtuální zařízení a modul uinput}

Zde popisujeme, jak nabídnout výstup slučovače k dispozici aplikacím.

Původní prototyp našeho slučovače je obsažen v souborech {\tt mice2mouse.\{cpp|h\}}.
Aplikace si pomocí tohoto modulu mohla zaregistrovat callback funkce volané
při nějaké události. Nicméně musela také zajistit pravidelné volání funkce
{\tt m2m\_workHorse()}, která zjišťovala, zdali je nějaká událost k dispozici
a příp. volala registrované funkce. Kompletní aplikace využívající
tento modul je v souboru {\tt main.cpp}.

Tento přístup není ideální, pro jsme hledali jiné řešní a to nalezli
v modulu linuxového jádra s názvem {\bf uinput}. Tento modul umožňuje
vytvořit virtuální vstupní zařízení a z userspace ho \uv{krmit} událostmi.
Implementace slučovače využívající služeb tohoto modulu je v souboru
{\tt m2m\_device.c}.

Výsledný spustitelný program má 2 povinné parametry\,--\,názvy speciálních
souborů zařízení pro myši {\bf A} a {\bf B} a jeden volitelný parametr
({\tt -m}) udávající mapování, tedy {\it mapstring}. Výchozí mapování
je {\tt xyxz}.

Poznámka: pro úspěšné spuštění je nutné mít v jádře zavedený modul {\tt uinput}.
To se dá zjistit pomocí příkazu {\tt lsmod | grep uinput} a případně
napravit pomocí {\tt modprobe uinput}.

Po spuštění se vytvoří virtuální zařízení (kde jinde než v {\tt /dev/input/})
s názvem \uv{m2m}. Toto zařízení, ač se tváří jako klasický joystick
(události typu {\tt EV\_ABS} pro 3 osy a nějaká tlačítka) ale ve skutečnosti
v události zasílá relativní odchylku. Toto je kvůli zjednodušení implementace.
Díky tomu, že je zařízení knihovnami jako GLUT či SDL rozpoznáno jako joystick,
nemusíme se o nic starat a pouze využívat standardních funkcí těchto knihoven
(např. v SDL funkce {\tt SDL\_JoyAxisEvent()}).

Pouze je nutné se při výběru joysticku dotazovat na joystick se jménem {\tt m2m}.
Toto by šlo v budoucnu jistě vyřešit čistěji.


\subsection{Použité knihovny}
Jak již bylo uvedeno, klientské aplikace mohou využít knihovny GLUT či
SDL, příp. jiné, které mají nějakou podporu pro joysticky.

Pokud chceme mít aplikaci mimální, může sama způsobem uvedeným výše
zpracovávat události ze souboru náležejímu zařízení m2m v {\tt /dev/input/}.

\subsection{Demonstrační aplikace}
Součástí výstupu projektu je implementace demonstračních aplikací využívající
naše rozhraní. Vytvořili jsme jednoduché {\bf intro}, dále aplikaci
pro {\bf kreslení ve 3D}, základ pro {\bf 3D piškvorky} a nakonec
aplikaci {\bf 3D moorhuhn}, která byla použita při testování rozhraní.
Pro postupné spuštění všech aplikací slouží příkaz {\tt make run}.

\clearpage
