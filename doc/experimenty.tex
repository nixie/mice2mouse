% File name: experimenty.tex
% Date:      13.12.2011 17:00

\section{Experimenty a testování}
Cílem experimentální části je získání užitečných informací o testovaném subjektu a z nich vycházející další vývoj. Tato kapitola bude tedy popisovat stategii testování a experimentů na rozhraní m2m, výběr vhodné skupiny účastníků těchto pokusů a popisem jednotlivých experimentů a analýzy jejich výsledků.

\subsection{Strategie experimentování a testování}
Úplně jako první bylo potřeba zjistit, jakým způsobem by měla být aplikace ovládána. Které tlačítka obou polohovacích zařízení používat ke kterým činnostem a jak správně nastavit osy pohybu každého ze vstupních zařízení. Jelikož má rozhraní pracovat v prostředí 3D, musí existovat prostředek pro prostorovou orientaci. Cílem druhého experimentu se tedy stalo zjištění reakcí uživatelů na různé možnosti natáčení 3D scény s ohledem na pohyb v 3D prostoru a jeho přizpůsobení. Získání výše zmíněných informací umožnilo vytvořit několik demonstračních aplikací, jež se později staly instruktážními nástroji pro poslední experiment. Než k němu ale došlo, bylo mezi účastníky testování provedena krítká analýza uživatelského rozhraní pro odstranění drobných nedostatků těchto jednoduchých aplikací. Cílem celého procesu experimentování a testování však bylo hlavně získání údajů o míře efektivnosti práce s takovým rozhraním. A právě to bylo námplní závěrečného experimentu. 

\subsubsection{Výběr účastníků experimentů}
Největší vliv na výběr skupiny testujících uživatelů byla jejich schopnost pracovat s počítačem, uživatel mající problém obsluhovat jedno polohovací zařízení bude jistě mít problém i se zařízením druhým. Proto jsme požádali o účast v experimentování skupinu 15-ti vysokoškolských studentů, kteří s požadavkem běžného ovládání počítače nemají problém.

\subsubsection{experiment č. 1 - orientace v prostoru}
Jak již bylo uvedeno výše, cílem prvního experimentu bylo nalezení optimálního ovládání a zjištění dalších možných nastavení, které by si eventuálně uživatel mohl explicitně zvolit. Jako forma získání informací byl zvolen dotazník %priloha dotaznik 1
. Z analýzy jeho výsledků vyplynulo, že: %dodelat

Pro další vývoj tedy bylo potřeba naimplementovat možnost záměny souřadných os mezi polohovacími zařízeními navzájem. Za zmínku také stojí fakt, že jak pravorukým, tak levorukým vyhovovalo maximálně prohození souřadných os, ovšem prohození významu levého a pravého tlačítka bylo vnímáno jako nepotřebné%opravdu?
Dotazník nakonec posloužil i k orientačnímu zisku informací o dalších aktivitách a zkušenostech účastníků v oblasti práce v 3D aplikacích.

experiment1 - pohyb v prostoru - zjistit pozadavky na ovladani
experiment2 - orientace v 3d prostoru
testování   - odstraneni drobnych nedostatku
experiment3 - schopnost prizpusobit se novemu prostredi za ucelem odhadu efektivity prace ve 3d prostoru pomoci rozhrani m2m

