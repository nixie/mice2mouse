% File name: navrh.tex
% Date:      13.12.2011 17:00

\section{Návrh}

Zařízení je navržené tak, aby jeho služeb mohl využívat úplně každý.
Hardwarové vybavení (tedy de facto pořizovací cena zařízení\,--\,software
je šířen v otevřené podobě pod GPL licencí) obnáší pouze 2 klasické myši.

\subsection{Slučovač}\label{merging}

Výstupem ze 2 myší jsou 4 nezávislé osy, které chceme nějak sloučit
do 3 a exituje mnoho kombinací, jak to udělat. Zařízení by mělo umožňovat
definici vlastního mapování, dle chuti každého.

Pro jednoznačné určení mapování budeme nadále používat tzv. {\it mapstring},
který určuje, která osa na kterou bude mapována. Myši rozlišujeme pomocí
písmen {\bf A} a {\bf B}.

Příklad mapstringu: {\tt xyxz}\,--\,osa $x$ myši A se mapuje na osu $x$, osa $y$ myši A se
mapuje na osu $y$, osa $x$ myši B se mapuje na osu $x$ a osa $z$ myši B se
mapuje na osu $z$.
V kapitole o testování (\ref{experiments}) se dozvíte, že mapování {\tt xzxy}
vyhovovalo většině lidí, na druhém místě pak skončilo mapování {\tt xyxz}.

\subsection{Fyzické uspořádání}

Při návrhu zařízení jsme zohlednili praktickou stránku, jak zacházet
se 2 myšmi:
\begin{itemize}
\item{{\bf Vše v 1 fyzické rovině}}
    \begin{itemize}
    \item[$+$]{ obě ruce stejně zatížené, žádná nemusí držet váhu myši a sebe u vertikálně položené podložky}
    \item[$-$]{ ne tak intuitivní, jako následující uspořádání}
    \end{itemize}
\item{{\bf Podložky obou myší fyzicky umístěny ortogonálně}\,--\,vlastnosti viz předchozí uspořádání, ale naopak}
\item{{\bf Varianta pro 1 ruku}\,--\,na klasické myši je zleva (myš pro praváky) umístěn drobný touchpad ovládaný palcem ovládající zbývající osu}

\end{itemize}
Nabízí se ještě možnost ovládat zbývající osu kolečkem myši, nicméně
tuto možnost jsme dále nezkoumali. Kolečka většiny myší mají navíc nedostačující
rozlišení (což by nesplňovalo podmínku, že chceme ovládat všechny 3 osy nezávisle
a rovnocenně).
\clearpage
\subsection{Existující zařízení}\label{solutions}

Obecný termín pro takováto zařízení je {\bf 3D myš}.

Na trhu se již dlouho vyskytují výrobky s podobnou funkcionalitou.
Jde např. od výrobky fy {\it 3Dconnexion}, kupř. {\bf SpaceNavigator}\,--\,zařízení
určené pro pohyb a minupulaci s objekty ve 3D. Toto zařízení umožňuje
ovládat 6 směrů pomocí jedné ruky (tažením, rotací, stlačením, vytažením
apod.).

Další možnosti:
\begin{itemize}
\item{výrobky fy {\bf Axsotic}\,--\,něco podobného, využívá standardu HID}
\item{{\bf Wii Remote Nunchuk}\,--\,obsahuje 3D akcelerometr a tlačítka}
\item{jistě by se našly i další}
\end{itemize}
