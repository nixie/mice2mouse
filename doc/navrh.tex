% File name: navrh.tex
% Date:      13.12.2011 17:00

\section{Návrh}

Zařízení je navržené tak, aby jeho služeb mohl využívat úplně každý.
Hardwarové vybavení (tedy de facto pořizovací cena zařízení\,--\,software
je šířen v otevřené podobě pod GPL licencí) obnáší pouze 2 klasické myši.

\subsection{Princip činnosti}

Princip činnosti navrhovaného zařízení je zobrazen na obrázku \ref{fig:basicidea}.

\begin{figure}[htb]
\centering
\includegraphics[width=0.6\textwidth]{img/basic_idea.eps}
\caption{Princip činnosti zařízení}
\label{fig:basicidea}
\end{figure}

Nejdůležitější část zařízení tvoří část {\bf Merging}, kde dochází k mapování
čtyřech vstupních os (osy $x$ a $y$, u obou myší) na osy 3 (odpovídající
osám 3D prostoru, ve kterém chceme pracovat). Implementace této části je
detailně popsána v sekci \ref{merging}.


\subsection{Fyzické uspořádání}

Při návrhu zařízení jsme zohlednili praktickou stránku, jak zacházet
se 2 myšmi:
\begin{itemize}
\item{{\bf Vše v 1 fyzické rovině}}
    \begin{itemize}
    \item[$+$]{ obě ruce stejně zatížené, žádná nemusí držet váhu myši a sebe u vertikálně položené podložky}
    \item[$-$]{ ne tak intuitivní, jako následující uspořádání}
    \end{itemize}
\item{{\bf Podložky obou myší fyzicky umístěny ortogonálně}\,--\,vlastnosti viz předchozí uspořádání, ale naopak}
\item{{\bf Varianta pro 1 ruku}\,--\,na klasické myši je zleva (myš pro praváky) umístěn drobný touchpad ovládaný palcem ovládající zbývající osu}
\end{itemize}


\subsection{Existující zařízení}\label{solutions}

Obecný termín pro takováto zařízení je {\bf 3D myš}.

Na trhu se již dlouho vyskytují výrobky s podobnou funkcionalitou.
Jde např. od výrobky fy {\it 3Dconnexion}, kupř. {\bf SpaceNavigator}\,--\,zařízení
určené pro pohyb a minupulaci s objekty ve 3D. Toto zařízení umožňuje
ovládat 6 směrů pomocí jedné ruky (tažením, rotací, stlačením, vytažením
apod.).

Další možnosti:
\begin{itemize}
\item{výrobky fy {\bf Axsotic}\,--\,něco podobného, využívá standardu HID}
\item{{\bf Wii Remote Nunchuk}\,--\,obsahuje 3D akcelerometr a tlačítka}
\item{jistě by se našly i další}
\end{itemize}
